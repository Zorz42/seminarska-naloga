\section{Optimizacija z dinamičnim programiranjem}\label{sec:optimizacija-z-dinamicnim-programiranjem}
Da bi pohitrili našo rešitev, bomo nekaj opazili:
Če privzamemo, da se globalne spremenljivke ne spreminjajo, potem je vrednost funkcije \texttt{najmanjsi\_cas} za neko stopnjo vedno enaka.
To pomeni, da ko enkrat končamo s tekom funkcije, lahko shranimo rezultat in pri naslednjem klicu funkcije preverimo, če smo že kdaj računali vrednost za to stopnjo.
Če smo, potem lahko kar vrnemo že izračunano vrednost, sicer pa izračunamo vrednost in jo shranimo.
To naredimo tako, da hranimo dve tabeli: eno za vrednosti, ki smo jih že izračunali in drugo, ki pove, ali smo že izračunali vrednost za neko stopnjo.
Potem pa ugotovimo naslednje dejstvo:
namesto, da bi funkcijo rekurzivno klicali, bi kar brez funkcije izpolnili tabelo z vrednostmi.
Ta optimizacija sicer ne spremeni časovne zahtevnosti, vendar pa bistveno pohitri izvajanje programa, saj so klici funkcij počasnejši od navadnih operacij.
\\
\\
Časovna zahtevnost te rešitve je $O(n^2)$, saj moramo za vsako stopnjo iti skozi vse naslednje stopnje, kar je že veliko bolje kot prejšnjih $O(2^n)$.zahtevnost