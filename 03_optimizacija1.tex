\section{Optimizacija z dinamičnim programiranjem}\label{sec:optimizacija-z-dinamicnim-programiranjem}
Da bi pohitrili našo rešitev, bomo nekaj opazili:
Če privzamemo, da se globalne spremenljivke ne spreminjajo, potem je vrednost funkcije \texttt{najmanjsi\_cas} za neko stopnjo vedno enaka.
To pomeni, da ko enkrat končamo s tekom funkcije, lahko shranimo rezultat in pri naslednjem klicu funkcije preverimo, če smo že kdaj računali vrednost za to stopnjo.
Če smo, potem lahko kar vrnemo že izračunano vrednost, sicer pa izračunamo vrednost in jo shranimo.
To naredimo tako, da hranimo dve tabeli: eno za vrednosti, ki smo jih že izračunali in drugo, ki pove, ali smo že izračunali vrednost za neko stopnjo.
Potem pa ugotovimo naslednje dejstvo:
namesto, da bi funkcijo rekurzivno klicali, bi kar brez funkcije izpolnili tabelo z vrednostmi.
Ta optimizacija sicer ne spremeni časovne zahtevnosti, vendar pa bistveno pohitri izvajanje programa, saj so klici funkcij počasnejši od navadnih operacij.
\\
\\
Časovna zahtevnost te rešitve je $O(n^2)$, saj moramo za vsako stopnjo iti skozi vse naslednje stopnje, kar je že veliko bolje kot prejšnjih $O(2^n)$.

\begin{lstlisting}[label={lst:code3}, language=C++]
// standardna knjiznica
#include<iostream>
#include<vector>
// definiramo krajsnjico za 64-bitno stevilo
typedef long long int64;
// da lahko opustimo "std::" predpono
using namespace std;


int n; // stevilo posasti
vector<int> moc_posasti; // moc vsake posasti
vector<int> spretnost_po; // nasa spretnost po uboju vsake posasti
vector<int64> najmanjsi_cas; // tabela namesto funkcije

int main(){
    // ni tako pomembno:
    // preberemo n in x
    int zacetna_spretnost;
    cin>>n>>zacetna_spretnost;
    // nastavimo velikost obeh seznamov na n
    moc_posasti.resize(n);
    spretnost_po.resize(n);
    // preberemo vrednosti in jih vnesemo v oba seznama
    for(int&i:moc_posasti)
        cin>>i;
    for(int&i:spretnost_po)
        cin>>i;

    // rezultat celega programa
    int64 rezultat = 1e18;

    // enaka koda kot prej, vendar namesto funkcije shranjujemo vrednosti v tabelo
    najmanjsi_cas.resize(n);
    najmanjsi_cas[n - 1] = 0;

    // zelo je pomembno, da ko racunamo stopnje, gremo od zadnje do prve,
    // saj pri racunanju uporabimo rezultate poznejsih stopenj
    for(int stopnja = n - 2; stopnja >= 0; stopnja--) {
        int trenutna_spretnost = spretnost_po[stopnja];

        int64 rezultat = 1e18;

        for(int naslednja_stopnja = stopnja + 1; naslednja_stopnja < n; naslednja_stopnja++) {
            int64 trenutna_vrednost = (int64) trenutna_spretnost * moc_posasti[naslednja_stopnja] + najmanjsi_cas[naslednja_stopnja];

            rezultat = min(rezultat, trenutna_vrednost);
        }

        najmanjsi_cas[stopnja] = rezultat;
    }

    // prvi uboj je lahko kjerkoli, gremo skozi vse mozne
    for(int prvi_uboj = 0; prvi_uboj < n; prvi_uboj++) {
        int64 trenutna_vrednost = 0;

        // to je cas, ki je potreben, da ubijemo tam bivajoco posast
        trenutna_vrednost += (int64) zacetna_spretnost * moc_posasti[prvi_uboj];

        // poklicemo funkcijo, da nam ugotovi koliko casa bo trajalo, da pridemo do konca
        trenutna_vrednost += najmanjsi_cas[prvi_uboj];

        rezultat = min(rezultat, trenutna_vrednost);
    }

    // rezultat bo na koncu vseboval najkrajsi cas
    // izpisi ga
    cout << rezultat << "\n";

    return 0;
}
\end{lstlisting}