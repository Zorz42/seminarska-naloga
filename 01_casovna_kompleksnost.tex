\section{Časovna kompleksnost}\label{sec:casovna-kompleksnost}
Preden se lotimo reševanja problema, si oglejmo, kaj je časovna kompleksnost in kako je uporabna:\\
Velik problem pride pri tem, da ni čas izvedbe vedno popolnoma enak, včasih je hitrejši, včasih počasnejši.
Zelo je odvisno od drugih procesov in preostalih okoliščin, zato opišemo čas izvajanja bolj približno.
Namesto, da konkretno povemo, koliko časa potrebuje algoritem, povemo, kako se čas izvajanja spreminja, ko se velikost vhoda spreminja.
Večina časa je ta ocena dovolj dobra, saj hitra rešitev porabi desetkrat ali pa celo stokrat manj časa, kot je na voljo.
Počasna resitev pa porabi lahko potencialno tudi tisočkrat več časa, kot je na voljo, zato je dovolj groba ocena.\\
Primer:
\begin{itemize}
    \item Algoritem, ki gre skozi seznam in izpiše vse elemente, ima časovno kompleksnost $O(n)$, kjer je $n$ dolžina seznama.
    \item Algoritem, ki gre skozi seznam in izpiše vse pare elementov, ima časovno kompleksnost $O(n^2)$, kjer je $n$ dolžina seznama.
    \item Algoritem, ki uredi seznam po vrsti, ima časovno kompleksnost $O(n\log n)$, kjer je $n$ dolžina seznama.
    Dokaz časovne kompleksnosti urejanja bomo tukaj opustili, saj ni pomemben del dokumenta.
\end{itemize}
Velikokrat je časovna kompleksnost le zgornja meja in se včasih izkaže, da je algoritem hitrejši, kot je bila ocena.
To novo oceno se da matematično dokazati a je velikokrat precej zapleteno.\\
Časovno kompleksnost ima tudi matematično definicijo:
\begin{definition}
    Naj bo $f$ funkcija, ki sprejme naravno število $n$ in vrne realno število.
    Časovna kompleksnost algoritma je $O(f(n))$, če obstajata pozitivni konstanti $c$ in $n_0$, da velja:
    \begin{equation*}
        \forall n \geq n_0: \text{čas izvajanja algoritma} \leq c \cdot f(n)
    \end{equation*}
\end{definition}
Ta definicija izgleda precej zapletena, a je v resnici precej preprosta.
Razmislek o tej definiciji je prepuščen bralcu.\\
