\section{Problem}\label{sec:problem}
Oglejmo si naslednji problem:
\begin{definition}
    Igraš igro, kjer imaš $n$ različnih stopenj.
    Vsaka stopnja ima neko pošast, ki ima določeno moč in se lahko odločimo, da jo premagamo ali pa preskočimo.
    Premagovanje pošasti nam vzame $s_i \cdot f$ časa, kjer je $s_i$ moč pošasti in $f$ naša spretnost (pozor: nižja kot je spretnost, manj časa potrebujemo).
    Če pošast preskočimo, nam to ne vzame časa.
    Ko premagamo pošast, se nam spretnost nastavi na $f_i$.
    Cilj igre je, da premagamo zadnjo pošast v čim manj časa.
    Napiši program, ki dobi $n$ - število stopenj, $s_i$ - moč pošasti na $i$-ti stopnji in $f_i$ - spretnost, ki jo dobimo, ko premagamo $i$-to pošast in izpiše najmanjši čas, ki ga potrebujemo, da premagamo zadnjo pošast.
\end{definition}
Ena izmed možnih rešitev je, da gremo skozi vse možnosti in izberemo najboljšo.
Čeprav je ta rešitev pravilna, je časovno precej neučinkovita in nam ne bo prinesla veliko točk na tekmovanju.
Časovna kompleksnost te rešitve je $O(2^n)$, ker je na vsaki stopnji dve možnosti: premagamo ali preskočimo pošast.\\
Precej enostaven način za spisat tako rešitev je rekurzija: definiramo funkcijo $dobi(int stopnja, int spretnost)$, ki nam vrne najmanjši čas, ki ga potrebujemo, da premagamo vse pošasti od stopnje $stopnja$ naprej, če imamo spretnost $spretnost$.
